\documentclass{beamer}
\usepackage[T1]{polski}
\usepackage[polish]{babel}
\usepackage[utf8]{inputenc}
\usepackage[T1]{fontenc}
\usepackage[mediumspace,mediumqspace,Grey,squaren]{SIunits}
\usepackage{graphicx}
\usepackage{listings}
\usepackage{color}
\usepackage[backend=biber]{biblatex}
\usepackage{filecontents}
\bibliography{\addbibresource{bibliography.bib}}


\definecolor{mygreen}{rgb}{0,0.6,0}
\definecolor{mygray}{rgb}{0.5,0.5,0.5}
\definecolor{mymauve}{rgb}{0.58,0,0.82}

\lstset{ %
	backgroundcolor=\color{white},   % choose the background color; you must add \usepackage{color} or \usepackage{xcolor}; should come as last argument
	basicstyle=\footnotesize,        % the size of the fonts that are used for the code
	breakatwhitespace=false,         % sets if automatic breaks should only happen at whitespace
	breaklines=true,                 % sets automatic line breaking
	captionpos=b,                    % sets the caption-position to bottom
	commentstyle=\color{mygreen},    % comment style
	deletekeywords={...},            % if you want to delete keywords from the given language
	escapeinside={\%*}{*)},          % if you want to add LaTeX within your code
	extendedchars=true,              % lets you use non-ASCII characters; for 8-bits encodings only, does not work with UTF-8
	frame=single,	                   % adds a frame around the code
	keepspaces=true,                 % keeps spaces in text, useful for keeping indentation of code (possibly needs columns=flexible)
	keywordstyle=\color{blue},       % keyword style
	language=Octave,                 % the language of the code
	morekeywords={*,...},           % if you want to add more keywords to the set
	numbers=left,                    % where to put the line-numbers; possible values are (none, left, right)
	numbersep=5pt,                   % how far the line-numbers are from the code
	numberstyle=\tiny\color{mygray}, % the style that is used for the line-numbers
	rulecolor=\color{black},         % if not set, the frame-color may be changed on line-breaks within not-black text (e.g. comments (green here))
	showspaces=false,                % show spaces everywhere adding particular underscores; it overrides 'showstringspaces'
	showstringspaces=false,          % underline spaces within strings only
	showtabs=false,                  % show tabs within strings adding particular underscores
	stepnumber=2,                    % the step between two line-numbers. If it's 1, each line will be numbered
	stringstyle=\color{mymauve},     % string literal style
	tabsize=2,	                   % sets default tabsize to 2 spaces
	title=\lstname                   % show the filename of files included with \lstinputlisting; also try caption instead of title
}

\graphicspath{ {images/} }

\begin{document}
\title{Interfejs mózg-komputer}   
\author{Jakub Postępski} 
\date{\today} 

\frame{\titlepage} 


\section{Wstęp}
\frame{
	\frametitle{Zastosowanie}
	
	\begin{itemize}
		\item Sterowanie robotami
		\item Egzoszkielety
		\item Komunikacja
		\item Przekazywanie sygnałów z komputera i czujników do mózgu
		\item Obejście problemu uszkodzonych nerwów
		\item Rehabilitacja
		\item Diagnostyka medyczna
		\item Wariografy
		\item Rozrywka
		\item Biofeedback
	\end{itemize}
}

\frame{
	\frametitle{Historia} 
	
	\begin{itemize}
		\item 1924 - Hans Berger - rejestracja fal alfa, przewody pod czaszką
		\item 1987 - Odkrycie sygnału neuronowego
		\item 1998 - Philip Kennedy i  Roy Bakay - pierwszy dobrej jakości implant mózgu u człowieka		
		\item 1999 - Yang Dan - zdekodowanie sygnału odpowiedzialnego za widzenie przez kota
		\item 1999 - Pierwsze małpy sterują ramieniem robota
		\item 2002 - Ślepy facet kieruje na dwójce		
		\item 2005 - Człowiek steruje sztuczną ręką
		\item 2008 - Zdekodowano wizję człowieka w miarę dobrej jakości
		\item 2015 - Połączono mózgi trzech rezusów
	\end{itemize}
}

\frame{
	\frametitle{Widzenie kota}
	\begin{figure}[h]
		\centering
		\includegraphics[width=0.7\textwidth]{catvision1}
	\end{figure}
	\begin{figure}[h]
		\centering
		\includegraphics[width=0.7\textwidth]{catvision3}
	\end{figure}
}

\frame{
	\includegraphics[width=\textwidth]{ramie}
}

\section{Budowa mózgu}
\frame{
	\frametitle{Mózg}
	Podstawową komórką generującą prąd jest neuron.
	\begin{itemize}
		\item\textbf{Płat czołowy }kora motoryczna, decyzje o działaniu, kora ruchowa, kora przedruchowa
		\item\textbf{Płat skroniowy }słuch, węch
		\item\textbf{Płat ciemieniowy } temperatura, czucie, ból, napięcia mięśni, 
		\item\textbf{Płat potyliczny }widzenie i skojarzenia z nim związane, przestrzeń
		\item\textbf{Móżdżek }koordynacja ruchowa, utrzymanie równowagi
		\item\textbf{Pień mózgu }funkcje życiowe
	\end{itemize}
}

\frame{
	\frametitle{Neuron}
	\includegraphics[scale=0.3]{neuron}
}

\frame{
	\begin{figure}[h]
		\centering
		\includegraphics[scale=0.6]{mozg3}
	\end{figure}
}

\frame{
	\frametitle{Dodatki do mózgu}
	
	\begin{itemize}
		\item\textbf{Komórki glejowe} Utrzymują neurony
		\item\textbf{Opona pierwsza }Odżywianie mózgu
		\item\textbf{Opona druga }W niej znajduje się płyn mózgowo-rdzeniowy, wyrównuje ciśnienie w czaszce i amortyzuje wstrząsy
		\item\textbf{Opona trzecia i czaszka }Ochrona
	\end{itemize}
}

\frame{
	\frametitle{Fale mózgowe}
	Fale mózgowe, wyglądają inaczej dla ludzi w róznym wieku, poziomie wykształcenia i płci
	\begin{itemize}
		\item\textbf{Delta: }1-4Hz, rejestrowane podczas snu, ew podczas intensywnego wysiłku umysłowego\linebreak
		\includegraphics[scale=0.3]{faledelta}
		\item\textbf{Theta: }4-7Hz, medytacja, hipnoza, marzenie, emocje, ból
		\includegraphics[scale=0.3]{faletheta}
	\end{itemize}
}

\frame{
	\frametitle{Fale mózgowe}
	\begin{itemize}
		\item\textbf{Alfa: }7-13Hz, pochodzą głównie z płatów potylicznych, stan czuwania, stan relaksu\linebreak	
		\includegraphics[scale=0.3]{falealfa}	
		\item\textbf{Beta: }12-28Hz, gotowość, zwyczajna codzienna aktywność, percepcja zmysłowa, głównie w okolicach potylicznych
		\includegraphics[scale=0.3]{falebeta}	
		\item\textbf{Gamma: }30-70Hz, procesy poznawcze, pamięć, wiązanie wielu różnych bodźców zmysłowych
	\end{itemize}
}

\section{Urządzenia wykorzystywane do BCI}
\frame{
	\frametitle{BCI inwazyjne}
	Urządzenia inwazyjne są wszczepiane bezpośrednio do mózgu pacjenta
	\begin{itemize}
		\item Dobra dokładność pomiarów
		\item Lepsza możliwość stymulacji
		\item Duże ryzyko powikłań
	\end{itemize}
}

\frame{
	\frametitle{BCI nieinwazyjne}
	Urządzenia nieinwazyjne nie wymagają żadnych operacji. Urządzenia częściowo inwazyjne wymagają operacji, jednak elektrody kładzione są na mózgu pacjenta, bądź sklepieniu czaszki.
	\begin{itemize}
		\item Gorsze pomiary, duże zakłócenia w EEG
		\item Praktycznie brak możliwości stymulacji
		\item Małe koszty
		\item Nie ma potrzeby zapewnienia pomocy medycznej
		\item Wymagane szkolenia
	\end{itemize}
}
\frame{
	\frametitle{Implant ślimakowy}
	Wszczepiany dzieciom i dorosłym z głuchotą lub głębokim niedosłuchem.
	\begin{figure}[h]
		\includegraphics[width=0.8\textwidth]{Cochearimplants}
	\end{figure}
	
}
\frame{
	\begin{figure}
		\includegraphics[width=0.8\textwidth]{cochlearimplant}
	\end{figure}
}

\frame{
	\frametitle{MEG - Magnetoencefalografia}
	\begin{itemize}
		\item Rejestruje pole magnetyczne wytwarzane przez mózg
		\item Wykorzystywane do określenia funkcji poszczególnych częsci mózgu
		\item Potrzeba ok 50000 neuronów, aby sygnał był mierzalny
	\end{itemize}
}

\frame{
	\frametitle{fMRI - Funkcjonalny Rezonans Magnetyczny}
	\begin{itemize}
		\item Mierzymy zmiany przepływu krwi i utlenowania aktywnej okolicy mózgu.
		\item Mózg umieszczamy w polu magnetycznym o równoległych liniach pola, i atakujemy go impulsami magnetycznymi o określonej częstotliwości
		\item Impulsy pola powodują wzbudzenie spinów protonów w jądrach wodoru
		\item W wyniku impulsów jądra atomów zostają namagnesowane i same zaczynają emitować pole magnetyczne.
	\end{itemize}
}

\frame{
	\includegraphics[width=\textwidth]{fmri}
}

\frame{
	\frametitle{EEG - Elektroencefalograf}
	\begin{itemize}
		\item Mierzymy napięcia generowane przez mózg.
		\item Mierzone sygnały mają napięcia rzędu kilkudziesięciu {\micro\volt}
		\item Zakłócenia z otoczenia i ludzkiego ciała
		\item Napięcie odkładające się na granicy skóry z elektrodą
		\item Do 256 elektrod na całej czaszce
	\end{itemize}
}
\frame{
	\includegraphics[width=\textwidth]{eeg1}
}

\frame{
	\frametitle{Elektrody}
	\begin{itemize}
		\item Elektrody mają swój opór i pojemność
		\item Mamy elektrody: miseczkowe, grzybkowe, węglowe, klipsowe, igłowe
		\item Powszechnie stosowane są elektrody mokre
	\end{itemize}
	\includegraphics[width=\textwidth]{elektroda1}
}

\frame{
	\begin{figure}[h]
		\centering
		\includegraphics[width=0.7\textwidth]{elektroda2}
	\end{figure}
}

\frame{
	\frametitle{Typowe rozmieszczenie elektrod}
	\begin{figure}[h]
		\includegraphics[width=0.7\textwidth]{rozmieszczenie}
	\end{figure}	
}

\frame{
	\frametitle{Przykładowe badanie EEG}
	\includegraphics[width=\textwidth, height=\textheight]{eeg4}
}

\section{BCI}
\frame{
	\frametitle{P300}
	Test polegający na prezentowaniu rzadko występujących bodźców, przykładem jest \textit{oddball}.\linebreak
	P3 lub P3b:
	\begin{itemize}
		\item Bodziec \textit{standard} często, \textit{target} rzadko
		\item Po pojawieniu się \textit{target} mamy wychylenie napięcia na elektrodach
		\item Aktywność głównie obszarów ciemieniowych
	\end{itemize}
	P3a:
	\begin{itemize}
		\item Dodatkowo bodziec \textit{novel}, który ma być ignorowany przez pacjenta
		\item Również mamy aktywność, lecz przesuniętą w stronę płatów czołowych
		\item Pojawia się minimalnie wcześniej
	\end{itemize}
	U osób młodszych można zaobserwować bardziej zróżnicowane rozłożenie napięć.
}


\frame{
	\frametitle{Obróbka sygnału EEG}
	Obowiązkowo zakładamy filtr na częstotliwość 50Hz.\linebreak
	W początkowej fazie odszumiamy sygnał:
	\begin{itemize}
		\item Filtry cyfrowe
		\item Filtry przestrzenne
		\item Separacja częstotliwości
	\end{itemize}
	Następnie uwidaczniamy to co nam potrzebne:
	\begin{itemize}
		\item Transformata Fouriera
		\item Transformata falkowa
		\item Statystyki wyższych rzędów
	\end{itemize}
	Na koniec, algorytm do wykrywania informacji:
	\begin{itemize}
		\item Modele autoregresyjne
		\item Sztuczne sieci neuronowe
		\item Liniowa analiza dyskryminacyjna
		\item Drzewa decyzyjne
	\end{itemize}
	Polecam program do nagrywania EEGStudio\par
	Polecam pakiet matlaba do obróbki EEGLab
}

\frame{
	\frametitle{FT vs WT}
	Transformata Fouriera nie pozwala na pozyskanie informacji o czasie zdarzenia.
	
	FT:
	\begin{itemize}
		\item Informacje o występujących w sygnale częstotliwościach 
		\item Jądrem są funkcje sinusoidalne
		\item Mamy jeden podstawowy wzór
		\item  $\hat f(\xi) = \int_{-\infty}^{\infty} f(x)e^{-2 \pi ix \xi}\, dx$
	\end{itemize}
	
	WT:
	\begin{itemize}
		\item Informacje o występowaniu częstotliwości w czasie
		\item Jądrem są funkcje falkowe
		\item Mamy rodzinę wzorów
		\item $X(a,b) = \frac{1}{\sqrt{a}}\int_{-\infty}^{\infty}\overline{\Psi\left(\frac{t - b}{a}\right)} x(t)\, dt$
	\end{itemize}
}

\begin{frame}[fragile]
	\frametitle{DWT}
	\lstset{language=Java}
	
	\begin{lstlisting}  % Start your code-block
	int[] output = new int[input.length];
	
	for (int length = input.length >> 1; ; length >>= 1){
	for (int i = 0; i < length; ++i){
	int sum = input[i * 2] + input[i * 2 + 1];
	int difference = input[i * 2] - input[i * 2 + 1];
	output[i] = sum;
	output[length + i] = difference;
	}
	if (length == 1){
	return output;
	}
	System.arraycopy(output, 0, input, 0, length << 1);
	}
	\end{lstlisting}
\end{frame}

\frame{
	\frametitle{Efekt WT}
	\begin{figure}
		\includegraphics[width=0.75\textwidth]{wt}
	\end{figure}	
}
\subsection{Zrób to sam}
\frame{
	\frametitle{Jak odebrać intencje użytkownika}
	\begin{itemize}
		\item Wyobrażanie sobie np figur przestrzennych
		\item Wyobrażanie sobie ruchu
		\item Naturalne myślenie
		\item Mruganie, ruszanie gałkami ocznymi (oszukaństwo ?)
	\end{itemize}
	\begin{itemize}
		\item Dla każdego użytkownika usytuowanie elektrod może być różne.
	\end{itemize}
}

\frame{
	\frametitle{Jak zbudować własne BCI}
	Powołując się na artykuł \textsl{Cursor Control Using EEG Signals from Eye Movement Potentials} autorstwa Granta G. Connella, wystarczy mieć tylko komputer i proste EEG.\linebreak
	\begin{figure}[h]
		\caption{Patrzenie w górę}
		\centering
		\includegraphics[width=0.8\textwidth]{lookingup}
	\end{figure}
	\begin{figure}[h]
		\caption{Patrzenie w dół}
		\centering
		\includegraphics[width=0.8\textwidth]{lookingdown}
	\end{figure}
}

\frame{
	\frametitle{Jak zbudować własne BCI}
	\begin{figure}[h]
		\caption{Patrzenie w lewo}
		\centering
		\includegraphics[width=0.8\textwidth]{lookingleft}
	\end{figure}
	\begin{figure}[h]
		\caption{Patrzenie w prawo}
		\centering
		\includegraphics[width=0.8\textwidth]{lookingright}
	\end{figure}
}
\frame{
	\frametitle{Potrzebny sprzęt}
	\begin{figure}[h]
		\includegraphics[scale=0.2]{olimex}
	\end{figure}
	\begin{figure}[h]
		\includegraphics[scale=0.3]{schema3}
	\end{figure}
}

\end{document}
