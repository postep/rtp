\documentclass{beamer}
\usepackage[T1]{polski}
\usepackage[polish]{babel}
\usepackage[utf8]{inputenc}
\usepackage[T1]{fontenc}
\usepackage[mediumspace,mediumqspace,Grey,squaren]{SIunits}
\usepackage{graphicx}
\usepackage[backend=biber]{biblatex}
\usepackage{filecontents}
\bibliography{\addbibresource{bibliography.bib}}



\graphicspath{ {images/} }

\begin{document}
\title{Interfejsy wymiany danych czasu rzeczywistego}   
\author{Jakub Postępski} 
\date{\today} 

\frame{\titlepage} 


\section{Wstęp}
\frame{
	\frametitle{Definicja i zastosowanie}
	
	\begin{itemize}
		\item 
	\end{itemize}
}

\frame{
	\frametitle{Systemy operacyjne czasu rzeczywistego}
}

\frame{
	\frametitle{Budowa ogólna}
	\begin{itemize}
		\item Fizyczne urządzenia i magistrale
		\item Algorytmy transportu danych
		\item Sterownik systemu operacyjnego
		\item Aplikacja
	\end{itemize}
}


\section{Warstwa fizyczna}
\frame{
	\frametitle{Wstęp}
	\begin{itemize}
		\item Łącza analogowe
		\item Łącza cyfrowe
	\end{itemize}
	\begin{itemize}
		\item Zwielokrotnianie falowe
		\item Wykrywanie kolizji
	\end{itemize}
}

\subsection{Przewodowe}
\frame{
	\frametitle{Linia z otwartym kolektorem / drenem}
	
}
\frame{
	\frametitle{Para różnicowa}
}
\frame{
	\frametitle{Magistrale szeregowe i równoległe}
}

\subsection{Światłowody}
\frame{
	\frametitle{Światłowody}
	\begin{itemize}
		\item Małe tłumienie
		\item Brak zakłóceń elektromagnetycznych
		\item Problemy z fizycznymi połączeniami
	\end{itemize}
	\begin{itemize}
		\item Wielomodowe
		\item Jednomodowe
	\end{itemize}
}

\subsection{Bezprzewodowe}
\frame{
	\frametitle{Mechanizm wykrywania kolizji}
}

\subsection{Ralizacje przewodowe}

\frame{
	\frametitle{CAN}
}

\frame{
	\frametitle{USB}
}

\frame{
	\frametitle{SPI}
}


\frame{
	\frametitle{UART}	
}


\frame{
	\frametitle{RS-232}	
}

\frame{
	\frametitle{I$^2$C}
}


\frame{
	\frametitle{Ethernet}
}

\section{Warstwa transportu danych}

\frame{
	\frametitle{Modbus}
}

\frame{
	\frametitle{CANOpen}
}

\frame{
	\frametitle{Ethercat}	
}

\frame{
	\frametitle{Gigavision}	
}


\section{Przykłady praktyczne}
\frame{
	\frametitle{Mała szklarnia}	
}

\frame{
	\frametitle{Serwonapęd}	
}

\frame{
	\frametitle{Czujnik IMU}	
}

\frame{
	\frametitle{Robot mobilny}
}

\frame{
	\frametitle{Kontrola trakcji elektrycznej lub kolejowej}	
}
\frame{
	\frametitle{Kamera HD}	
}

\end{document}
